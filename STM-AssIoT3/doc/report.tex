\documentclass[a4paper]{article}

\usepackage[english]{babel}
\usepackage[utf8]{inputenc}
\usepackage{hyperref}
\usepackage{float}
\usepackage{todonotes}
\usepackage[margin=1.5cm]{geometry}
\usepackage{graphicx}
\usepackage{tikz}
\usetikzlibrary{automata, positioning, arrows}

\author{Luca Casadei - 0001069237\\Francesco Pazzaglia - 0001077423}
\date{Last modified: \today}
\title{\textbf{Assignment nr.3\\Towards IoT: Smart Temperature Monitoring}}

\begin{document}
	\maketitle
	\tableofcontents
	
	\section{Introduction}
	This document describes a smart temperature monitoring system consisting of four integrated subsystems:
	
	\begin{itemize}
		\item \textbf{Temperature Monitoring Subsystem} (ESP32): Continuously samples temperature data and communicates via MQTT;
		\item \textbf{Control Unit} (Java/Vert.x): Acts as system brain handling state management, data aggregation, and inter-subsystem coordination;
		\item \textbf{Window Controller} (Arduino): Manages physical window actuator and operator interface with LCD display;
		\item \textbf{Dashboard} (Python/Tkinter): Provides GUI for remote monitoring and control;
	\end{itemize}
	
	The system features two operational modes: \textit{AUTOMATIC} with temperature-dependent window control, and \textit{MANUAL} for direct operator intervention. Communication uses MQTT for sensor data, HTTP for dashboard updates, and serial protocol for actuator control.
	
	
\end{document}