\documentclass[a4paper]{article}

\usepackage[english]{babel}
\usepackage[utf8]{inputenc}
\usepackage{hyperref}
\usepackage{float}
\usepackage{todonotes}
\usepackage[margin=1.5cm]{geometry}

\author{Luca Casadei - 0001069237\\Francesco Pazzaglia - 0001077423\\Mattia Morri - 0001082027}
\date{Last modified: \today}
\title{\textbf{Assignment nr.2\\FSMs and Task-based Architectures: Smart Waste Disposal System}}

\begin{document}
	\maketitle
	\tableofcontents
	\section{Introduction}
	This project implements a Smart Waste Disposal System for liquid waste using an Arduino-based embedded system and a PC application for monitoring and operator control. It is based on a finite state machine (FSM) and task-based architecture.
	
	\section{Finite State Machine and Task Descriptions}
	The Arduino project developed through PlatformIO has the objective to manage the whole system through a change of states. In particular, we planned to divide the solution into 3 main tasks, which are:
	\begin{enumerate}
		\item \textbf{WasteTask}\\
		Manages waste disposal operations, the main task for the proper functioning of the entire system. It provides for the following states:
		\begin{itemize}
			\item \textbf{CLOSED}: the neutral state of the system.
			\item \textbf{OPEN}: when a user wants to throw something away he presses the open button and opens the door.
			\item \textbf{RECEIVED}: reception of waste.
			\item \textbf{ERRORED}: if the temperature increases too much
			\item \textbf{FULL}: when the container fills up, so capacity goes to zero.
			\item \textbf{EMPTYING}: when the operator empties the container.
		\end{itemize}
		
		\item \textbf{SleepingTask}\\
		Manages the shutdown of the system when no user is detected in the proximity of the waste container. It provides for the following states:
		\begin{itemize}
			\item \textbf{AWAKE}: the system is in normal mode of operation.
			\item \textbf{SLEEPING}: if no users were found nearby.
		\end{itemize}
		
		\item \textbf{TemperatureTask}\\
		Manages the temperature changes of the waste inside the container. It provides for the following states:
		\begin{itemize}
			\item \textbf{STABLE}: the temperature is below the maximum threshold.
			\item \textbf{UNSTABLE}: the temperature is above the maximum threshold for a temporarily period.
 			\item \textbf{DANGER}: the temperature has been above the threshold for too long.
		\end{itemize}

	\end{enumerate}

	\section{Dashboard for Monitoring and Control}
	The Operator Dashboard, developed in Java, provides real-time monitoring and operator control for the system. The GUI interacts with the Arduino system via a serial connection, displaying the container's current status and allowing user actions.

	\subsection{Features}
	\begin{itemize}
		\item \textbf{Visual Representation of Waste Levels}: 
		The GUI includes a container panel, which dynamically updates to show the percentage of waste in the container.
		\item \textbf{Real-time Updates}: 
		Serial communication allows the GUI to receive data from the Arduino system, such as waste level and temperature status.
		\item \textbf{User Interaction}: 
		Buttons are provided to control the container, including:
		\begin{itemize}
			\item \textbf{Empty the Container}: Decreases the waste level progressively when pressed.
			\item \textbf{Restore the Container}: Resets the container waste level.
		\end{itemize}
	\end{itemize}

\end{document}
